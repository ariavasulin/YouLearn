\documentclass[../master/master.tex]{subfiles}

\begin{document}

\section{Resources}

\begin{lecturesummary}
\textbf{Course Resources:} Textbooks, supplementary materials, and references for Math 104. This section is updated throughout the course as new resources are discovered.
\end{lecturesummary}

\subsection{Primary Textbook}

\begin{summarybox}
\textbf{Walter Rudin}, \textit{Principles of Mathematical Analysis}, 3rd edition. McGraw-Hill, 1976. ISBN: 978-0-07-054235-8.

The standard reference for undergraduate real analysis. Chapters 1--7 are covered in this course. Known for its concise, rigorous style. Expect to read proofs multiple times.
\end{summarybox}

\subsection{Supplementary Materials}

\begin{itemize}
    \item \textbf{Abbott}, \textit{Understanding Analysis}, 2nd ed. --- More accessible introduction. Good for building intuition before tackling Rudin.
    \item \textbf{Tao}, \textit{Analysis I \& II} --- Builds analysis from the ground up. Excellent for students who want to see every detail.
    \item \textbf{Pugh}, \textit{Real Mathematical Analysis} --- Beautiful exposition with great exercises and pictures.
\end{itemize}

\subsection{Online Resources}

\begin{itemize}
    \item Francis Su's \textit{Real Analysis} lecture series (Harvey Mudd, YouTube) --- Exceptional lectures covering Rudin chapter by chapter.
    \item MIT OCW 18.100A --- Problem sets and lecture notes for a similar course.
\end{itemize}

\end{document}
