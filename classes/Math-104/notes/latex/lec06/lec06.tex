\documentclass[../master/master.tex]{subfiles}

\begin{document}

\section{Lecture 6: Introduction to Group Theory}

\begin{lecturesummary}
Group Theory is the study of algebraic structures known as groups, which are sets equipped with an operation that satisfies certain axioms. This lecture will cover the definition of a group, examples of groups, and key properties. 
\end{lecturesummary}

\subsection{Definition of a Group}
\begin{summarybox}
A group is a set $G$ combined with an operation \ast that satisfies the following properties:
\begin{itemize}[nosep]
    \item \textbf{Closure:} For all $a, b \in G$, the result of the operation $a \ast b$ is also in $G$.
    \item \textbf{Associativity:} For all $a, b, c \in G$, $(a \ast b) \ast c = a \ast (b \ast c)$.
    \item \textbf{Identity Element:} There exists an element $e \in G$ such that for every element $a \in G$, $e \ast a = a \ast e = a$.
    \item \textbf{Inverse Element:} For each $a \in G$, there exists an element $b \in G$ such that $a \ast b = b \ast a = e$.
\end{itemize}
\end{summarybox}

\subsection{Examples of Groups}
\begin{summarybox}
\begin{itemize}[nosep]
    \item The set of integers under addition, $(\mathbb{Z}, +)$, is a group.
    \item The set of non-zero rational numbers under multiplication, $(\mathbb{Q}^*, \cdot)$, is a group.
    \item The set of permutations of a finite set forms a group known as the symmetric group.
\end{itemize}
\end{summarybox}

\end{document}