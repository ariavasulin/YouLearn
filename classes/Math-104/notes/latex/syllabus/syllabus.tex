\documentclass[../master/master.tex]{subfiles}

\begin{document}

\section{Syllabus}

\begin{lecturesummary}
\textbf{Course Overview:} Math 104 is an introduction to real analysis, covering the rigorous foundations of calculus. We study the real number system, sequences and series, continuity, differentiation, and Riemann integration, with emphasis on proofs. The primary text is Walter Rudin's \textit{Principles of Mathematical Analysis} (3rd edition).
\end{lecturesummary}

\subsection{Course Information}

\begin{summarybox}
\textbf{At a Glance:}
\begin{itemize}[nosep]
    \item \textbf{Course}: Math 104 --- Introduction to Real Analysis
    \item \textbf{Term}: Spring 2026
    \item \textbf{Schedule}: MWF 10:00--10:50 AM, Evans Hall 3
    \item \textbf{Textbook}: Rudin, \textit{Principles of Mathematical Analysis}, 3rd ed.
    \item \textbf{Prerequisites}: Math 53 and Math 54 (or equivalents)
\end{itemize}
\end{summarybox}

\subsection{Course Requirements}

\begin{itemize}
    \item \textbf{Homework (40\%)}: Weekly problem sets from Rudin. Due Fridays. Lowest score dropped.
    \item \textbf{Midterm (25\%)}: In-class, Week 8. Covers Chapters 1--4.
    \item \textbf{Final (35\%)}: Comprehensive. Covers Chapters 1--7.
\end{itemize}

\subsection{Learning Objectives}

By the end of this course, students will be able to:
\begin{enumerate}
    \item Construct rigorous proofs involving properties of the real numbers.
    \item State and apply the completeness axiom (Least Upper Bound Property).
    \item Prove convergence or divergence of sequences and series.
    \item Define and prove properties of continuous functions on metric spaces.
    \item Apply compactness and connectedness in proofs.
    \item Develop and present mathematical arguments with clarity and precision.
\end{enumerate}

\subsection{Assignments}

\begin{summarybox}
\textbf{Homework Schedule:}
\begin{itemize}[nosep]
    \item \textbf{HW 1} --- Rudin Ch.\ 1: Problems 1, 5, 9, 18 (Bonus: 7, 8, 20) \hfill \textit{Due: Jan 31}
    \item \textbf{HW 2} --- Rudin Ch.\ 2: Problems 6, 22, 27, 29 (Bonus: Kuratowski) \hfill \textit{Due: Feb 7}
    \item \textbf{HW 3} --- Rudin Ch.\ 2--3: TBD \hfill \textit{Due: Feb 14}
    \item \textbf{HW 4} --- Rudin Ch.\ 3: TBD \hfill \textit{Due: Feb 21}
    \item \textbf{HW 5} --- Rudin Ch.\ 4: TBD \hfill \textit{Due: Feb 28}
\end{itemize}
\end{summarybox}

\subsection{Course Calendar}

\begin{center}
\begin{tabular}{|c|l|l|l|}
\hline
\textbf{Week} & \textbf{Dates} & \textbf{Topics} & \textbf{Reading/Due} \\
\hline
1 & Jan 20--24 & Ordered sets, LUB property, fields & Rudin 1.1--1.4 \\
2 & Jan 27--31 & Construction of $\R$, Archimedean property & Rudin 1.5--1.8; \textbf{HW 1 due} \\
3 & Feb 3--7 & Countability, metric spaces & Rudin 2.1--2.3; \textbf{HW 2 due} \\
4 & Feb 10--14 & Topology, open/closed sets & Rudin 2.4--2.6; \textbf{HW 3 due} \\
5 & Feb 17--21 & Compactness, Heine-Borel & Rudin 2.7--2.8; \textbf{HW 4 due} \\
6 & Feb 24--28 & Sequences, subsequences & Rudin 3.1--3.3; \textbf{HW 5 due} \\
7 & Mar 3--7 & Series, convergence tests & Rudin 3.4--3.7 \\
8 & Mar 10--14 & \textbf{Midterm (Wed)} & Covers Ch.\ 1--4 \\
\hline
\end{tabular}
\end{center}

\end{document}
